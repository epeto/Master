
\documentclass{article}
\usepackage[spanish]{babel} %Definir idioma español
\usepackage[utf8]{inputenc} %Codificacion utf-8
\usepackage{amssymb, amsmath, amsbsy, wasysym}
\usepackage{multirow} % para tablas
\usepackage{graphicx}
\usepackage{hyperref}
\usepackage{listings}
\usepackage[ruled, vlined, spanish, linesnumbered]{algorithm2e} %Para escribir algoritmos
\title{Propuesta de proyecto}
\author{Emmanuel Peto Gutiérrez}
\begin{document}
\maketitle

Se utilizará aprendizaje por refuerzo profundo para hacer control, aplicado al problema del péndulo invertido (\textit{cartpole}).

\begin{itemize}

\item \textbf{Referencia:}

Lillicrap, T. P., Hunt, J. J., Pritzel, A., Heess, N., Erez, T., Tassa, Y., Silver, D., and Wierstra, D. \textit{Continuous Control With Deep Reinforcement Learning}, Google Deepmind, 2016.

\item \textbf{Enlace al paper:}

\url{https://paperswithcode.com/paper/continuous-control-with-deep-reinforcement}

\item \textbf{Enlace al ambiente de prueba:}

\url{https://gymnasium.farama.org/environments/classic_control/cart_pole/}

\end{itemize}

\end{document}

