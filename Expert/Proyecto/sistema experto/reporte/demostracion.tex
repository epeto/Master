\documentclass{article}
\usepackage[spanish]{babel}
\usepackage[utf8]{inputenc}
\usepackage{amssymb, amsmath, amsbsy, wasysym}
\usepackage{multirow}
\usepackage{graphicx}
\usepackage{listings}
\title{Reporte de proyecto\\Sistemas expertos}
\author{Emmanuel Peto Gutiérrez}
\begin{document}
\maketitle

\section{Crear consulta}

La primera parte del sistema consiste en crear una consulta, dadas las restricciones del usuario.

\subsection{Incluir}

La primera restricción es: incluir un ingrediente

Ejemplos de los ingredientes que puede incluir:
\begin{itemize}
\item butter
\item salt
\item olive oil
\item honey
\end{itemize}

\subsection{Excluir}

Segunda restricción: excluir un ingrediente. Los ejemplos son los mismos que en la inclusion.

\subsection{Categoría}

Tercera restricción: elegir una categoría.

El usuario que compartió la receta le pone ciertas etiquetas (\textit{tags}) a una receta, que pueden describir el tipo de receta que está subiendo.

Ejemplos de categorías:
\begin{itemize}
\item north-american
\item breakfast
\item dietary
\item christmas
\end{itemize}

\subsection{Requerimientos nutricionales}

Cuarta restricción: contenido menor al promedio.

El usuario puede elegir que un componente de su receta sea menor al promedio, por ejemplo, que la receta sea baja en azúcares.

Los componentes nutricionales de las recetas son:
\begin{itemize}
\item calories
\item total\_fat
\item sugar
\item sodium
\item protein
\item saturated\_fat
\item carbohydrates
\end{itemize}

Cuando ya se eligieron las cuatro restricciones, se guardan todas las recetas que cumplan con esas condiciones en un archivo llamado \texttt{query\_id\_recipes.csv}

\subsection{Interacción}

A continuación se muestra el resultado de la interacción de un usuario con el sistema:

\begin{verbatim}
prompt:
Elija una acción:
1.-Crear consulta
2.-Crear plan
3.-Consultar ingredientes y pasos de un plan
usuario:
1

prompt:
¿Desea incluir ingredientes?
1.-Sí
2.-No
usuario:
1

prompt:
Ingrese la lista de ingredientes a incluir, separados solo por comas.
usuario:
bananas

prompt:
¿Desea excluir ingredientes?
1.-Sí
2.-No
usuario:
1

prompt:
Ingrese la lista de ingredientes a excluir, separados solo por comas.
usuario:
onion

prompt:
¿Las recetas pertenecen a alguna categoría?
1.-Sí
2.-No
usuario:
1

prompt:
Ingrese las categorías, separadas solo por comas.
usuario:
brunch

prompt:
¿Desea agregar requerimientos nutricionales?
1.-Sí
2.-No
usuario:
2

Recetas guardadas en archivo: query_id_recipes.csv

\end{verbatim}

Para este ejemplo, el contenido del archivo \texttt{query\_id\_recipes.csv} es: 9,84

Lo que quiere decir que solo las recetas 9 y 84 cumplen con los requerimientos del usuario.

\section{Crear plan}

Para esta parte debe existir el archivo \texttt{query\_id\_recipes.csv}, que contiene los ids con las restricciones del usuario.

La segunda parte del sistema\footnote{Hay que ejecutar el programa una vez por cada parte que se requiera.} consiste en crear un plan:

\begin{itemize}
\item Una receta: si se elige este plan, se tomarán 10 opciones aleatorias del archivo \texttt{query\_id\_recipes.csv} y se guardarán los nombres de receta en un archivo llamado \texttt{query\_una.csv}.
\item Un día: se tomarán 9 opciones aleatorias y se crearán 3 columnas: desayuno, comida y cena. Cada columna tendrá 3 opciones de recetas. Se guarda en un archivo: \texttt{query\_dia.csv}.
\item Una semana: se tomarán 15 opciones aleatorias y se crearán 5 columnas, una para cada día de la semana. Cada columna tendrá 3 recetas: una para el desayuno, una para la comida y una para la cena. Se guarda en un archivo \texttt{query\_semana.csv}.
\end{itemize}

Se hará una demostración de interacción con el usuario. Para esto, supondremos que el archivo \texttt{query\_id\_recipes.csv} tiene los ids: 9, 20, 25, 32, 57, 101, 201, 250, 300, 321, 400, 464, 480, 508, 645, 670, 712, 844, 911, 950.

\begin{verbatim}
prompt:
Elija una acción:
1.-Crear consulta
2.-Crear plan
3.-Consultar ingredientes y pasos de un plan
usuario:
2

prompt:
Elija qué tipo de plan requiere:
1.-Una receta
2.-Recetas para el día
3.-Plan semanal
usuario:
2

prompt:
9 recetas guardadas en: query_dia.csv
\end{verbatim}

El contenido del archivo \texttt{query\_dia.csv} es:

desayuno,comida,cena\\
300$|$you can t eat just one ice box cookies,57$|$make that chicken dance salsa pasta,911$|$11 secret herbs and spices kfc copycat\\
20$|$cream of spinach soup,250$|$party taco dip,321$|$bailey s hot fudge sauce\\
844$|$10 minute mustard dip,670$|$not meatballs,508$|$pumpkin pie filling for mexico

\section{Listar ingredientes y pasos}

Para esta parte se supone que existe al menos uno de los siguientes archivos:
\begin{itemize}
\item \texttt{query\_una.csv}
\item \texttt{query\_dia.csv}
\item \texttt{query\_semana.csv}
\end{itemize}

Una vez que ya se tienen guardados los ingredientes del plan, se despliegan los ingredientes y los pasos.

Una de las opciones es obtener la lista completa de ingredientes. Esto podría servirle al usuario para hacer la lista del supermercado e ir a comprar la despensa de la semana (en el caso del plan semanal).

Otra opción es elegir un id de receta para desplegar los ingredientes y luego los pasos. Antes de eso verá el \textit{dataframe} con la información de las recetas para saber qué \texttt{id} elegir.

Se muestra una interacción con el usuario.

\begin{verbatim}
prompt:
Elija una acción:
1.-Crear consulta
2.-Crear plan
3.-Consultar ingredientes y pasos de un plan
usuario:
3

prompt:
Ingrese el plan que quiere consultar:
1.-Una receta
2.-Un día
3.-Semanal
usuario:
2

prompt:
El número a la izquierda es el id de la receta
                                     desayuno  ...                                        cena
0  300|you can t eat just one ice box cookies  ...  911|11 secret herbs and spices kfc copycat
1                    20|cream of spinach soup  ...                321|bailey s hot fudge sauce
2                   844|10 minute mustard dip  ...          508|pumpkin pie filling for mexico
[3 rows x 3 columns]

prompt:
Elija una acción:
1.-Obtener lista completa de ingredientes
2.-Mostrar ingredientes y pasos de una receta
usuario:
2

prompt:
Ingrese el id de la receta que quiere obtener
usuario:
20

prompt:
Lista de ingredientes:
water
salt
boiling potatoes
fresh spinach leaves
unsalted butter
coarse salt
fresh ground black pepper
nutmeg

Pasos a seguir:
bring water and salt to a boil
cut the potatoes in half lengthwise
and then into 1 / 2 thick slices
boil over medium-high heat for twenty minutes
add the spinach and boil another 10 minutes being careful to not overcook in order to maintain its bright green color
drain
reserving 1 1 / 4 cups cooking liquid
in a food processor
process the potatoes and spinach until very smooth
adding the butter 1 / 2 t at a time
return the puree to the pot and add the reserved liquid until the desired consistency
reheat slowly and season with salt and pepper
ladle into shallow soup plates and garnish with nutmeg or shaved cheese
\end{verbatim}

\end{document}

