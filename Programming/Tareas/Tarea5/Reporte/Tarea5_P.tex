
\documentclass{article}
\usepackage[spanish]{babel} %Definir idioma español
\usepackage[utf8]{inputenc} %Codificacion utf-8
\usepackage{amssymb, amsmath, amsbsy, wasysym}
\usepackage{multirow} % para tablas
\usepackage{graphicx}
\usepackage[ruled, vlined, spanish, linesnumbered]{algorithm2e} %Para escribir algoritmos
\title{Tarea 5\\Programación avanzada}
\author{Emmanuel Peto Gutiérrez}
\begin{document}
\maketitle

\section*{MiClaseUtil}

\begin{verbatim}
/**
 * En esta clase se define el método estático: maxInArray, que recibe un arreglo
 * de un tipo primitivo numérico y devuelve el elemento máximo de ese arreglo.
 */
public class MiClaseUtil{
    public static byte maxInArray(byte[] arr){
        byte maximo = arr[0];
        for(int i=1; i<arr.length; i++){
            if(arr[i] > maximo){
                maximo = arr[i];
            }
        }
        return maximo;
    }

    public static short maxInArray(short[] arr){
        short maximo = arr[0];
        for(int i=1; i<arr.length; i++){
            if(arr[i] > maximo){
                maximo = arr[i];
            }
        }
        return maximo;
    }

    public static int maxInArray(int[] arr){
        int maximo = arr[0];
        for(int i=1; i<arr.length; i++){
            if(arr[i] > maximo){
                maximo = arr[i];
            }
        }
        return maximo;
    }

    public static long maxInArray(long[] arr){
        long maximo = arr[0];
        for(int i=1; i<arr.length; i++){
            if(arr[i] > maximo){
                maximo = arr[i];
            }
        }
        return maximo;
    }

    public static float maxInArray(float[] arr){
        float maximo = arr[0];
        for(int i=1; i<arr.length; i++){
            if(arr[i] > maximo){
                maximo = arr[i];
            }
        }
        return maximo;
    }

    public static double maxInArray(double[] arr){
        double maximo = arr[0];
        for(int i=1; i<arr.length; i++){
            if(arr[i] > maximo){
                maximo = arr[i];
            }
        }
        return maximo;
    }
}
\end{verbatim}

\section*{Inventario y producto}

\begin{verbatim}
public class Inventario {

    //Variable estática que lleva el conteo del siguiente id.
    private static int nextId = 99;

    //método que incrementa la variable de clase nextId y la devuelve.
    public static int getNextId(){
        nextId++;
        return nextId;
    }
}
\end{verbatim}

\begin{verbatim}
public class Producto {
    
    private int id; //identificador del producto

    //Constructor del producto
    public Producto(){
        id = Inventario.getNextId();
    }

    //Devuelve el id del producto
    public int getId(){
        return id;
    }

}
\end{verbatim}

\section*{Pregunta}

\textbf{¿Qué modificador permite que múltiples variables de una misma instancia sean compartidas a través de todos los objetos de una clase? Explica brevemente un poco sobre este.}

El modificador es \texttt{static}. Al poner este modificador en una variable, ésta se vuelve una variable de la clase y no de un objeto en particular. Si un objeto escribe sobre esa variable, entonces ese valor será el mismo independientemente del objeto a través del cual se acceda a la variable.

\end{document}

