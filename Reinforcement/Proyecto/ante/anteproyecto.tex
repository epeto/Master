
\documentclass{article}
\usepackage[spanish]{babel} %Definir idioma español
\usepackage[utf8]{inputenc} %Codificacion utf-8
\usepackage{amssymb, amsmath, amsbsy, wasysym}
\usepackage{multirow} % para tablas
\usepackage{hyperref}
\usepackage{graphicx}
\usepackage[ruled, vlined, spanish, linesnumbered]{algorithm2e} %Para escribir algoritmos
\title{Anteproyecto\\Aprendizaje por refuerzo}
\author{Emmanuel Peto Gutiérrez}
\begin{document}
\maketitle

\textbf{Categoría:} Reproducción.

\textbf{Integrantes:} Emmanuel Peto Gutiérrez.

\section{Problema}

\begin{itemize}

\item \textbf{¿Qué tarea o problema se estudiará?}

Se replicarán resultados de ``Tune PI Controller Using Reinforcement Learning''.

\item \textbf{¿Dónde se obtendrán los datos, simulador o sistema del mundo real?}

Los datos se obtendrán de un simulador.

\item \textbf{¿Cuál es la principal hipótesis que se investigará?}

Se comparará la efectividad del aprendizaje por refuerzo con un afinador de un sistema de control.

\item \textbf{¿Cómo se relaciona con RL?}

Se usará aprendizaje por refuerzo para encontrar los parámetros de un controlador.

\end{itemize}

\section{Metodología}

\begin{itemize}

\item \textbf{Describir la clase de métodos que se utilizarán (de acuerdo al curso)}

Se usará el método TD3

\item \textbf{¿Qué literatura se utilizará para evaluar los resultados? cualitativa y cuantitativa}

Se usarán los resultados encontrados en el siguiente link:

\url{https://la.mathworks.com/help/reinforcement-learning/ug/tune-pi-controller-using-td3.html}

\end{itemize}

\end{document}

