
\documentclass{article}
\usepackage[spanish]{babel} %Definir idioma español
\usepackage[utf8]{inputenc} %Codificacion utf-8
\usepackage{amssymb, amsmath, amsbsy, wasysym}
\usepackage{multirow} % para tablas
\usepackage{graphicx}
\usepackage[ruled, vlined, spanish, linesnumbered]{algorithm2e} %Para escribir algoritmos
\title{Ejercicio 1\\Inteligencia artificial}
\author{Emmanuel Peto Gutiérrez}
\begin{document}
\maketitle

Se hará resolución binaria hasta obtener la cláusula vacía con el siguiente conjunto de cláusulas:

\[\Gamma = \{\lnot A \lor B \lor C, A , \lnot B, \lnot C\}\]

\begin{itemize}
\item[1)] $\lnot A \lor B \lor C$
\item[2)] $A$
\item[3)] $\lnot B$
\item[4)] $\lnot C$
\item[5)] $B \lor C$, Res(1, 2)
\item[6)] $\lnot A \lor C$, Res(1, 3)
\item[7)] $\lnot A \lor B$, Res(1, 4)
\item[8)] $C$, Res(2, 6)
\item[9)] $\square$, Res(4, 8)
\end{itemize}

El cuadro $\square$ representa la cláusula vacía.

\end{document}

