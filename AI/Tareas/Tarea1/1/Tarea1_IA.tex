
\documentclass{article}
\usepackage[spanish]{babel} %Definir idioma español
\usepackage[utf8]{inputenc} %Codificacion utf-8
\usepackage{amssymb, amsmath, amsbsy, wasysym}
\usepackage{multirow} % para tablas
\usepackage{graphicx}
\usepackage[ruled, vlined, spanish, linesnumbered]{algorithm2e} %Para escribir algoritmos
\title{Tarea 1\\Inteligencia artificial}
\author{Emmanuel Peto Gutiérrez}
\begin{document}
\maketitle

\section*{Crítica al artículo \textit{Computing machinery and intelligence}}

En el artículo \textit{Computing machinery and intelligence}, Alan Turing empieza planteando la pregunta: ``¿Las máquinas pueden pensar?'' Pero la pregunta le parece ambigua y, en vez de adentrarse en esa pregunta, describe el \textit{Juego de la imitación} y se pregunta si, cuando una máquina toma parte de uno de los interrogados en el juego, el interrogador podría equivocarse el decidir cuál jugador es máquina y cuál es un humano.

Por supuesto, a Alan Turing no le interesan que las máquinas se parezcan físicamente a los humanos, sino las máquinas que piensen como humanos. Describe el tipo de máquina a participar en el juego como una \textit{máquina de estados discretos}, lo cual hoy conocemos como autómata. Para simplificar, reduce el problema a utilizar una computadora digital como la máquina, dejando claro que es equivalente a otras máquinas como un humano haciendo cálculos en un papel o al motor analítico de Charles Babbage, en el sentido de que los cálculos que pueden realizar son los mismos, aunque el tiempo de cómputo sea diferente.

Luego, habla de diferentes puntos de vista que tratan de negar que las computadoras puedan pensar, y también trata de argumentar por qué una máquina que pueda jugar al juego de la imitación no contradice los argumentos en contra de la pregunta ``¿Las computadoras pueden pensar?''

Menciona que el tipo de máquina que pueda jugar al juego de la imitación debe ser una \textit{máquina que aprenda} y hace una analogía con el cerebro de un niño, el cual puede aprender mediante la educación que ha recibido, teniendo un sistema de castigos y recompensas.

Alan Turing parecía optimista al decir que en unos 50 años (o sea, a principios de los 2000) sería común hablar de máquinas que piensan (probablemente Stanley Kubrik pensaba lo mismo). Aunque ha habido avances en el área de las máquinas que interactúan con humanos, yo no podría afirmar que las máquinas están pensando.

Parece que para Alan Turing era suficiente que una máquina jugara bien al juego de la imitación para considerar que pensaba como humano. Yo estoy en desacuerdo con ese pensamiento y pongo un ejemplo: actualmente interactuamos con bots en el soporte técnico en línea. Quizá algunos de ellos sean humanos y quizá otros sean bots, y algunas personas (del lado del entrevistador) no podrán distinguirlos. Para mí, esto no es suficiente para afirmar que la computadora del otro lado está ``pensando'', sino que tiene las respuestas en bases de datos porque la mayoría de las personas hacen preguntas similares.

\end{document}

